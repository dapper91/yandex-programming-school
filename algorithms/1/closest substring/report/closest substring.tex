\documentclass[12pt]{article}
\usepackage{amsmath}
\usepackage[utf8]{inputenc}
\usepackage[russian]{babel}
\usepackage{fullpage}


\title{\bf Домашнее задание №3}
\author{Д.А. Першин}
\date{\today}


\begin{document}
\maketitle


\section{Словесное описание алгоритма}

При решении данной задачи будем использовать алгоритм быстрого преобразования Фурье (временная слжность - $O(n\log n)$, память - $O(n\log n)$).\\
 Входные массивы $a$ и $b$ ($len(a)=n, len(b)=m, n\geq m$) разложим на 4 массива каждый следующим образом: $a_A$ будет содержать единицы там, где в исходном массиве $a$ находилось значение A, остальные нули; $a_C$ будет содержать единицы там, где в исходном массиве $a$ находилось значение С, остальные нули; аналогично для $a_G$ и $a_T$, а также для массива $b$. Дополним каждый из 8 массиво до одинаковой длины следующим образом: к массывам $a_A$, $a_C$, $a_G$ и $a_G$ справа допишим $m$ нулей, к массивам $b_A$, $b_C$, $b_G$ и $b_G$ - $n$ нулей (для верного расчета циклической корреляции). В итоге получаем 8 массивов $m+n$ длины каждый. Затем дополним длины массивов до ближайшей степени двойки (для выполнения быстрого преобразоваия Фурье).\\

Далее для каждой пары векторов, полученных из $a$ и $b$ с одинаковым индексом ищем циклическую корреляцию. По теореме о циклической корреляции (спектр циклической корреляции есть произведение спектров сигналов) для функций $f$, $f'$ и $f''$, где 

$$f = \frac{1}{N} \sum_{m=0}^{n-1} f'_m f''_{m-n}, n = 0,1,...,N-1 $$
$$D_n = D^{'*}_n D''_n, n = 0,1,...,N-1$$

Вычисление над функциями производится за $O(n^2)$, но для ДПФ последоветельностей данных функций $D'$, $D''$ циклическая корреляция вычисляется за $O(n)$, а алгоритм быстрого преобразования Фурье позволяет находить ДПФ последоветельнсти за $O(n\log n)$ (обратное пробразование также за $O(n\log n)$).\\

Далее поэлементно складываем массивы для найденных циклических корреляций. В итоге получаем массив $c$, каждый элемант $c_i$ которого показывает кол-во совпадающих элементов массива $b$ и массива $a$ начиная с индекса $i$. Поиск максимального элемента дает искомый индекс сдвига в массиве $a$ при наложении на него массива $b$ (также следует проверить не выходит ли массив $b$ за границу массива $a$, так как изначально мы увеличили его длину на $m$ и дополнили до ближайшей степени 2). Если таких индексов несклько, то выбираем минимальный.




\paragraph{Алгоритм:}

\begin{enumerate}

\item 2 входных массива разложем на 8 битовых массива, как описано выше.

\item Дополним каждый из массивов до одинаковой длины $l_a = l_b = m + n $, а затем до ближайшей степени двойки, остаток заполнив 0.

\item Найдем циклическую корреляцию для каждой пары массивов $a$ и $b$ с одинаковым индексом $A$, $C$, $G$ или $T$ используя алгоритм БПФ, а затем ОБПФ, назовем их $c_A$, $c_C$, $c_G$, $c_T$.

\item Сложим получившиеся массивы поэлементно $C[i] = c_A[i] + c_C[i] + c_G[i] + c_T[i]$.

\item Найдем максимум в массиве $C$, где $C[j] = \smash{\displaystyle\max_{i \leq n-m} C[i]}$, $j$ выберем минимальным из всех, удовлетворяющих данному условию.

\item $j$ - искомый индекс.
	
\end{enumerate}
	



\section{Доказательство корректнсти}
Предположим, что найденный индекс $j$ не является индексом для наложения с максимальным совпадением, но в таком случае существует другой индекс $j'$, такой что при наложении $b$ на $a[j']$ получается большее количество совпадений, чем  для $j$, но это противоречит теореме о циклической корреляции.
Таким образом индекс $j$ является индексом, таким что при наложении последовательности $b$ на $a[j]$ получается максимальное количество поэлементных совпадений.


\section{Асимптотические оценки}
В результате получаем сложность по памяти $O((n+m) \log (n+m))$, так как мы используем 2 массива длины $m$ и $n$, 8 массивов длиной $n+m$, рекурсивный алгоритм БПФ требует $O((n+m) \log (n+m))$ дополнителной памяти. Cложность по времени равна $O((n+m)\log (n+m))$, так как мы используем БПФ и ОБПФ - $O((n+m)\log (n+m))$, поиск циклической корреляции для ДПФ последоветельностей за $O((n+m))$, поиск максимального элемента в результирующем массиве за $O(n-m)$.



\end{document}