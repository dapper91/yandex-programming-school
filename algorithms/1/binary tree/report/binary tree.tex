\documentclass[12pt]{article}
\usepackage{amsmath}
\usepackage[utf8]{inputenc}
\usepackage[russian]{babel}
\usepackage{fullpage}


\title{\bf Домашнее задание №4}
\author{Д.А. Першин}
\date{\today}


\begin{document}
\maketitle


\section{Словесное описание алгоритма}

Пусть имеется входной массив $a$ длины $n$, содержащий вершины бинарного дерева поиска в preorder порядке. Необходимо вывести дерево в inorder и postorder порядке. Будем использовать следующее утверждение: первая вершина $a_1$ во воходной последовательности являеся корнем (очевидно из определения preorder порядка). Для данной вершины (корня) существует левое и правое поддеревья $a_l = [a_2,a_{k-1}]$ и $a_r = [a_k, a_{n-1}]$, найдем индекс $k$. Из определения бинарного дерева поиска очевидно, что это будет первая вершина, большая или равная (из условия) чем корень (такое $k$: $a_k >= a_1$). Для поиска $k$ будем испльзовать алгоритм \emph{All nearest greater values} (будет описан ниже).

В итоге получаем индексы корня, правого поддерева, левого поддерева. Для вывода дерева в inorder порядке выводим сначало левое поддерево, затем текущий корень, затем правое поддерево, для postorder - левое поддерево, правое поддерево, корень. Данный алгритм выполняестя рекурсивно для правого и левого поддерева, пока поддерево не будет состоять из одного элемента без поддеревьев (листа). В таком случае просто выводим лист.\\

Алгоритм \emph{All nearest greater values} - алгоритм поиска ближайшего првого соседа, большего или равного, чем текущий. Алгоритм работает следующим образом: начинаем обход массива с конца, создадим стек $s$, а также результирующий массив $r$. Если текущий элемент $a_i$ менше или равен элемету на вершине стека $s$, то для текущего элемента он будет большем или равным ближайшим соседом, $r[i] = k$, где $k$ - индекс элемента на вершине стека, далее кладем текущий элемент в стек. Если текущий элемент больше, чем элемент на вершине стека, начинаем извлекать значания из стека, пока не найдем большее или равное значение и выполняем предыдущий пункт, или пока стек не опустеет, тогда для текущего элемента не существует такого соседа - $r[i] = None$, кладем текущий элемент в стек. Алгоритм выполняеся для $i = n-1...0$.
Несложно заметить, что каждый элемент кладется в стек и извлекается не более одого раза, следовательно временная сложность $O(n)$; если массив - строго возрастающая последовательность, то глубина стека может достигнуть $n$, следовательно сложность по памяти - $O(n)$.


\paragraph{Алгоритм:}

\begin{enumerate}

\item Для каждого элемента $a_i$вохдного массива $a$ длиный $n$ найдем ближайшего большего или равного соседа:

	\begin{enumerate}
	
	\item создадим пустой стек $s$ (вершина - $top(s)$, глубина - $size(s)$), результирующий массив $r$ длины $n$, занчание $a_i$ будем записывать в стек как пару чисел $(i, a_i)$;
	
	\item для всех $a_i$ пока $size(s) \neq 0$ и $a_i > top(s)$, извлекаем значение из стека;

	\item если  $size(s) = 0$, то $r_i = None$, кладем $a_i$ на стек;\\
		  иначе $r_i = index(top(s))$, кладем $a_i$ на стек;
		 
	\item переходим к пункту $b$
	
	\end{enumerate}
	

\item $left = 0$, $right = n - 1$.

\item Для массива $[a_{left}, a_{right}]$ $a_{left}$ - корень поддерева, $pivot = r_{left}$.

\item 	\begin{itemize}

		\item если $left = right$, то:\\
		$print(a_{left})$\\
		$return$

		\item для вывода $inorder$ порядка:\\
		рекурсивно пункт 3 для $left = left + 1$, $right = pivot - 1$\\
		$print(a_{left})$\\
		рекурсивно пункт 3 для $left = pivot$, $right = right$\\
		$return$
		
		\item для вывода $postorder$ порядка:\\
		рекурсивно пункт 3 для $left = left + 1$, $right = pivot - 1$\\		
		рекурсивно пункт 3 для $left = pivot$, $right = right$\\
		$print(a_{left})$\\
		$return$
				
		\end{itemize}
	
\end{enumerate}
		
	

\section{Доказательство корректнсти}
Предположим, что элемент $a_{left}$ не является корнем текущего поддерева, но данное условие противоречит определению $preorder$ порядка (сначала печатаеся корень поддерева, а затем правое и левое поддеревья), следовательно $a_{left}$ действительно является корнем текущего поддерева. Предположим, что $p = r_i$ не является первой вершиной правого поддерева, тогда существует вершина $k > p$, такая что $a_k < a_{left}$, но данное условие противоречит определению бинарного дерева поиска, следовательно $r_i$ является первой вершиной правого поддерева.



\section{Асимптотические оценки}
В результате получаем сложность по памяти $O(n)$, так как мы используем воходной массив длины $n$, стек для поиска ближайшего большего или равного соседа, глубина которого может достинать в среднем $\log (n)$, в худшем случае -  $n$, а также рекурсивный алгоритм для вывода массива в inorder и postorder порядке, глубина рекурсии также в среднем - $\log (n)$, в худшем случае - $n$. Cложность по времени равна $O(n)$, так как мы используем рекурсивный алгоритм вывода дерева - в среднем $O(\log (n))$, в худшем случае - $O(n)$, поиск ближайшего большего или равного соседа - $O(n)$.



\end{document}