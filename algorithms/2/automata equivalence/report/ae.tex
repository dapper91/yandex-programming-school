\documentclass[12pt]{article}
\usepackage[utf8]{inputenc}
\usepackage[russian]{babel}
\usepackage{fullpage}
\usepackage{caption}

\usepackage{amsmath}
\usepackage{algorithm}
\usepackage{algpseudocode}

\title{\bf Домашнее задание №2}
\author{Д.А. Першин}
\date{\today}


\begin{document}
\maketitle


\section{Словесное описание алгоритма}

Даны два детерминированных конечных автомата A и B. Необходимо определить, эквивалентны ли они. Для этого построим автомат, состояния которого равны объединению состояний двух автоматов A и B. Далее проверим эквивалентность начальных состояний автоматов A и B в объедененном автомате. Если состояния эквивалентны (неразличимы), то автоматы эквивалентны, иначе неэквивалентны. Эквивалентность состояний будем проверять с помощью алгоритма заполнения таболицы неэквивалентности.\\

\textit{Алгоритм заполнения таболицы неэквивалентности}:\\
Состоянием $\delta(s,w)$ назовем состояние, в которое автомат переходит из состояния $s$ по цепочке $w$.
Два состояния $s_1$ и $s_2$ являются эквивалентными, если для всех входных цепочек $w$ состояние $\delta(s_1,w)$ является допускающим тогда и только тогда, когда состояние $\delta(s_2,w)$ - допускающее.\\
Таблица неэквивалентности это таблица размера $n\times n$, где $n$ - кол-во состояний в автомате, на перечечении строки $i$ и столбца $j$ которой находится 1, если состояния неэквиваленты (различимы), иначе 0. Заполнять таблицу будем следующим образом: во первых различимыми являются пары состояний, одно их которых является терминальным (допустимым), а другое нет, отметим их. Далее различимыми являются состяния, которые переходят в различимые состояния по одним и тем же символям алфавита. Для этого построим список обратных ребер (переходов) для каждого состояния и будем двигаться от уже различимых состояний по обратным ребрам, отмечая их тоже как различимые. Процесс будем проводить итеративно, используя некоторый аналог поиска в ширину (добовляя вновь полученные различимые состояния в очередь и отмечая все состояния, достижимые из текущего состояния по обратным ребрам).\\

Кратко опишем алгоритм проверки эквивалентности автоматов:

\begin{enumerate}

	\item построим автомат $C = A + B$.

	\item построим таблицу неэквивалентности для автомата $C$.

	\item определим различимы ли начальные состояния автоматов A и B в объедененном автомате $C$.
	
	\begin{itemize} 
	
		\item если состояния различимы, то автоматы не эквивалентны.

		\item если состояния неразличимы, то автоматы эквивалентны.	
	
	\end{itemize}

\end{enumerate}



\paragraph{Алгоритм:}

\begin{itemize}

\item Пусть $a$ - начальное состояние автомата $A$, $b$ - начальное состояние $B$, $M$ - таблица неэквивалентности объедененного автомата C (алгоритм 2). Определим являются ли автоматы эквивалентными:

\captionof{algorithm}{проверка эквивалентности автоматов}
\begin{algorithmic}[1]	
	\Procedure{IsEquivalent}{$A,B$}
		\State $C=A+B$\Comment{$C$ - объединение автоматов $A$ и $B$}
		\State $M = NonequivalenceTable(C)$
		\If{$M[a][b] = true$}\Comment{если состояния различимы}
			\State return false\Comment{автоматы $A$ и $B$ неэквивалентны}
		\Else 
			\State return true\Comment{автоматы $A$ и $B$ эквиваленины}
		\EndIf			
	\EndProcedure
\end{algorithmic}	



\item Создадим пустую очередь из пар неэквивалентных состояний Q, M - искомая таблица неэквивалентности, $\delta^{-1}(i,j)$ - список обратных ребер (алгоритм 3), T - список терминальных вершин автомата $C$, $\Sigma$ - алфавит. Построим таблицу неэквиватентности:

\captionof{algorithm}{построение таблицы неэквивалентности}
\begin{algorithmic}[1]
	\Procedure{NonequivalenceTable}{$C$}
		\For{$\forall i \in [0, n-1]$}
			\For{$\forall j \in [0, n-1]$}
				\If{$M[i][j] = false$ and $T[i] \neq T[j]$}\Comment{одно состояние терминальное, другое нет}
					\State $M[i][j] = M[j][i] = true$\Comment{пометить состояния как неэквив.}
					\State $Q.push(<i,j>)$\Comment{добавить пару вершин в очередь}
				\EndIf
			\EndFor
		\EndFor\\
		\State $\delta^{-1} = InverseEdges(C)$\\
		
		\While{$!Q.empty()$}\Comment{помечаем состояния, достижимые из различимых состояний по обратным ребрам как различимые}
			\State $<u, v> = Q.poll()$
				\For{$c \in \Sigma$}
					\For{$r \in \delta^{-1}[u][c]$}
						\For{ $s \in \delta^{-1}[v][c]$}
							\If{$M[r][s] == false$}
								\State $M[r][s] = M[s][r] = true$
								\State $Q.push(<r, s>)$
							\EndIf							
						\EndFor					
					\EndFor
				\EndFor
		\EndWhile\\
	\State return $M$
	\EndProcedure
\end{algorithmic}	



\item Обозначим через $C$ автомат, через $T(s,t)$ матрицу переходов автомата (переход из состояния $s$ по символу $t$), $\delta^{-1}(s,t)$ - искомый список обратных ребер для состояния $s$ по символу $t$. Построим список обратных ребер:

\captionof{algorithm}{построение списков обратных ребер}
\begin{algorithmic}[1]
	\Procedure{InverseEdges}{$C$}
		\For{$\forall s \in C$}\Comment{для всех состояний автомата $C$}
			\For{$\forall t \in s$}\Comment{для всех переходов из состояния $s$}
				\If{$T(s,t) \neq None$}\Comment{если переход существует}
					\State $\delta^{-1}(T(s,t), t)).push(s)$\Comment{добавить обратное ребро}
				\EndIf
			\EndFor
		\EndFor\\
	\State return $\delta^{-1}$
	\EndProcedure
\end{algorithmic}	

	
\end{itemize}
	



\section{Доказательство корректнсти}
Корректность алгоритмов объединения автоматов и построения списков обратных ребер достаточно очивидна. Докажем корректность алгоритма построения таблицы неэквивалентности. Доказательство будем проводить по индукции. Пусть $s_1$ и $s_2$ - состояния, для которых существует символ $c$, приводящий их в различимые состояния $s_{1}' = \delta(s_1,c)$ и $s_{2}' = \delta(s_2,c)$. Тогда существует цепочка $w$, различающая их, т.е существуют различимые состояния $\delta'(s_{1}',w)$ и $\delta'(s_{2}',w)$, одно из которых является допускающим, а другое нет. Но в этом случае цепочка $cw$ отличает состояния $s_1$ и $s_2$, т.к. $\delta(s_1,cw)$ и $\delta(s_2,cw)$ - различимы, следовательно $s_1$ и $s_2$ также различимы.\\
Теперь докажем, что если начальные состояния автоматов в объединенном автомате различимы, то автоматы неэквивалентны. Предположим, что начальные состояния оказались различимы. В этом случае существует входная цепочка $w$, которая переводит одина автомат в допустимое состояние, а другой автомат - в недопустимаое, но это и является критерием неэквиватентности автоматов. Если же состояния неразличимы, то по любой входной цепочке оба автомата переходят либо в допустимое состояние, либо в недопустипое, а это и есть критерий их эквивалентности.




\section{Асимптотические оценки}
Пусть суммарное кол-во состояний в автоматах $A$ и $B$ равно $n$, длина алфавита $\Sigma$ (по условию $\Sigma \leq 26$). В результате получаем сложность по памяти $O(n^2)$, так как мы используем матрицу переходов размера $n\times\Sigma$, вектор терминальных сосотояний длины $n$, таблицу неэквивалентности размера $n\times n$, очередь неэквивалентных состояний, максимальная длина которой не превышает $n$ и списки обратных ребер максимальной длины, равной кол-ву ребер ($n\times\Sigma$).\\
Cложность по времени равна $O(n^2)$, так как для объединения автоматов требуеся линейное время $O(n)$, для построения списков обратных ребер требуется $O(n\times\Sigma)$ (необходимо перебрать все ребра), а для построения таблицы неэквивалентности требуется $O(n^2)$ ($O(n^2)$ в первом цикле для пометки и добавления в очередь всех пар состояний, одно из которых терминальное, а другая нет, и $O(n\times\Sigma)$ для обхода оставшихся состояний по спискам обратных ребер).\\
Итого, получаем временную сложность $O(n^2)$, сложность по памяти $O(n^2)$



\end{document}