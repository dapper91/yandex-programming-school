\documentclass[12pt]{article}
\usepackage{amsmath}
\usepackage[utf8]{inputenc}
\usepackage[russian]{babel}
\usepackage{fullpage}


\title{\bf Домашнее задание №1}
\author{Д.А. Першин}
\date{\today}


\begin{document}
\maketitle


\section{Словесное описание алгоритма}
Дана строка $s$. Требуется вычислить префикс-функцию $s$ — для каждого $i$ от 1 до |$s$|.
Считать значения префикс-функции $\pi[i]$ будем по очереди для всех $i$ от 1 до $n-1$. $\pi[0]$ присвоим зачение 0.
Для подсчёта текущего значения $\pi[i]$ заведем переменную $len_p$, обозначающую длину текущего рассматриваемого образца. На каждой итерации шага $len_p$ = $\pi[i-1]$ (для шага 1 $len_p$ = 0). Далее начинаем тестирвать образец длины $len_p$, для чего сравниваем символы $s[len_p]$ и $s[i]$. Если они совпадают, то $\pi[i]$ = $len_p+1$ и переходим к следующей итерации $i = i+1$. Если же символы отличаются, то уменьшаем длину $len_p$, присваевая ей значение $\pi[len_p-1]$, и повторяем этот шаг алгоритма с начала. Если $len_p=0$ и при этом так и не нашлость совпадения, присваеваем $\pi[i]$ = 0 и переходим к следующей итерации $i = i + 1$.






\paragraph{Алгоритм:}

\begin{enumerate}

\item cоздадим массив $\pi$ длины $n$. $\pi[0]$ = 0.

\item для всех $i \in [1, n-1]$:

	\begin{itemize} 
	
	\item $len_p = \pi[i-1]$.

	\item пока $len_p > 0$ и $s[len_p] \neq s[i]$: $len_p$ = $\pi[len_p-1]$.	
	
	\item если $s[i] = s[len_p]$: $len_p  = len_p + 1$.

	\item $\pi[i] = len_p$.
	
	\end{itemize}
	
\end{enumerate}
	



\section{Доказательство корректнсти}
$\pi[0] = 0$ по определению префикс-функции. Далее доказательство будем строить по индукции. Пусть мы имеем $s[0,len_p-1]$ = $s[i-1-len_p, i-1]$ и $s[i]$ = $s[len_p]$. В этом случае $len_p$ = $len_p + 1$ и $\pi[i] = len_p$, что очевидно верно, т.к. префикс и суффикс увеличились на 1 символ вправо, при этом элементы $s[i]$ и $s[len_p]$ равны, следовательно и равны новый префикс и суффикс.\\
В случае же, когда $s[i] \neq s[len_p]$, $len_p = \pi[len_p]$, что также верно, т.к. в этом случае мы должны найти наиболее длинный префикс ($len_p$), для которого выполняется свойство $s[0,len_p]$ = $s[i-len_p, i]$, который очевидно можно найти из ранее расчитанных значений префикс функции $\pi$, т.е. $len_p$ = $\pi[len_p]$ и снова выполняем проверку $s[i] == s[len_p]$. Если данное условие верно, то доказательство аналогично пункту 1, если нет - то шагу 2.




\section{Асимптотические оценки}
В результате получаем сложность по памяти $O(n)$, так как мы используем два массива $s$ и $\pi$ длиной $n$.Cложность по времени равна $O(n)$, так как указатель на правую границу $i$ проверяемой подстроки $s$ на каждом шаге увеличивается не более чем $n$ раз (длина самой строки) или остается неизменным. Если $i$ остается неизменным, то уменьшается $len_p$ (длина текущий проверяемого образца), которая увеличивается только при увеличении $i$ (т.е. любое уменьшение $len_p$ было обеспечено увеличением $i$ ранее), следовательно в сумме не может превышать $i$. Итого получаем временную сложность не более $2n$ или $O(n)$



\end{document}